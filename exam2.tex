\documentclass[10pt]{article}
\usepackage{amsmath, amssymb, geometry}
\usepackage{listings}
\usepackage{xcolor}
\geometry{margin=0.5in}
\lstset{
  language=R,
  basicstyle=\ttfamily\footnotesize,
  frame=single,
  breaklines=true,
  columns=fullflexible,
  showstringspaces=false
}
\begin{document}
\noindent
\begin{tabular}{ll}
$\bar{x}$: Sample mean, $\bar{x}=\dfrac{1}{n}\sum_{i=1}^n x_i$ & $x_i$: Individual observation \\[0.5em]
$s$: Sample standard deviation & $\sigma$: Population standard deviation \\[0.5em]
$n$: Sample size & $\mu_0$: Hypothesized mean under $H_0$ \\[0.5em]
$t_{\alpha/2,\,n-1}$: Critical value from the $t$-distribution & $z_{\alpha/2}$: Critical value from the Normal distribution \\[0.5em]
$\hat{p}$: Sample proportion & $p_0$: Hypothesized proportion \\[0.5em]
$\operatorname{Cov}(X,Y)$: Covariance, $E[(X-\mu_X)(Y-\mu_Y)]$ & $\rho_{XY}$: Correlation, $\dfrac{\operatorname{Cov}(X,Y)}{\sigma_X\sigma_Y}$ \\
CLT: Central Limit Theorem (implies $\bar{X}\sim N(\mu,\sigma^2/n)$ for large $n$)\\[1em]
\end{tabular}
% One-Sample t-Test (unknown sigma)

\noindent\textbf{1. One-Sample t-Test (unknown $\sigma$):} \\
Test statistic: $t=\dfrac{\bar{x}-\mu_0}{s/\sqrt{n}}$. \quad CI: $\bar{x}\pm t_{\alpha/2,n-1}\dfrac{s}{\sqrt{n}}$. 

 

 
\begin{lstlisting}
ciOneSampleT <- function(xbar, s, n, conf.level = 0.95) {
  alpha <- 1 - conf.level
  tcrit <- qt(1 - alpha/2, df = n - 1)
  margin <- tcrit * s / sqrt(n)
  c(xbar - margin, xbar + margin)
}
\end{lstlisting}

 

% One-Sample z-Test (known sigma)
\noindent\textbf{2. One-Sample z-Test (known $\sigma$):} \\
Test statistic: $z=\dfrac{\bar{x}-\mu_0}{\sigma/\sqrt{n}}$. \quad CI: $\bar{x}\pm z_{\alpha/2}\dfrac{\sigma}{\sqrt{n}}$.

 

 
\begin{lstlisting}
ciOneSampleZ <- function(xbar, sigma, n, conf.level = 0.95) {
  alpha <- 1 - conf.level
  zcrit <- qnorm(1 - alpha/2)
  margin <- zcrit * sigma / sqrt(n)
  c(xbar - margin, xbar + margin)
}
\end{lstlisting}

 

% Distribution of the Sample Mean
\noindent\textbf{3. Distribution of Sample Mean:} \\
By CLT, $\bar{X}\sim N\left(\mu,\dfrac{\sigma^2}{n}\right)$.

 

% Paired t-Test
\noindent\textbf{4. Paired t-Test:} \\
Let $D_i=X_i-Y_i$. Then, $\bar{D}=\dfrac{1}{n}\sum_{i=1}^{n}D_i$, \quad
$s_D=\sqrt{\dfrac{1}{n-1}\sum_{i=1}^{n}(D_i-\bar{D})^2}$, \quad 
$t=\dfrac{\bar{D}}{s_D/\sqrt{n}}$ \quad (CI: $\bar{D}\pm t_{\alpha/2,n-1}\dfrac{s_D}{\sqrt{n}}$).

 

 
\begin{lstlisting}
pairedTTestCI <- function(x1, x2, conf.level = 0.95, alternative = "two.sided") {
  diff <- x1 - x2
  n <- length(diff)
  dbar <- mean(diff)
  s_diff <- sd(diff)
  SE <- s_diff / sqrt(n)
  t_stat <- dbar / SE
  if (alternative=="two.sided")
    p_val <- 2 * min(pt(t_stat, df = n-1), 1 - pt(t_stat, df = n-1))
  else if (alternative=="less")
    p_val <- pt(t_stat, df = n-1)
  else if (alternative=="greater")
    p_val <- 1 - pt(t_stat, df = n-1)
  CI <- t.test(x1, x2, paired = TRUE, conf.level = conf.level, alternative = alternative)$conf.int
  list(n = n, mean_diff = dbar, s_diff = s_diff, SE = SE, t_statistic = t_stat,
       p_value = p_val, CI = CI)
}
\end{lstlisting}

 

% Confidence Interval for a Proportion
\noindent\textbf{5. Confidence Interval for a Proportion:} \\
$\hat{p}\pm z_{\alpha/2}\sqrt{\dfrac{\hat{p}(1-\hat{p})}{n}}$.

 

 
\begin{lstlisting}
ciProp <- function(phat, n, conf.level = 0.95) {
  alpha <- 1 - conf.level
  zcrit <- qnorm(1 - alpha/2)
  margin <- zcrit * sqrt(phat * (1 - phat)/n)
  c(phat - margin, phat + margin)
}
\end{lstlisting}

 

% One-sample Proportion Test
\noindent\textbf{6. One-Sample Proportion Test:} \\
Test statistic: $z=\dfrac{\hat{p}-p_0}{\sqrt{\frac{p_0(1-p_0)}{n}}}$; \quad p--value (two-sided): $2\bigl[1-\Phi(|z|)\bigr]$.

 

 
\begin{lstlisting}
oneSamplePropTest <- function(x, n, p0, alternative = "two.sided") {
  phat <- x / n
  SE <- sqrt(p0*(1 - p0)/n)
  z_stat <- (phat - p0) / SE
  if(alternative=="two.sided") {
    p_val <- 2 * (1 - pnorm(abs(z_stat)))
  } else if(alternative=="less") {
    p_val <- pnorm(z_stat)
  } else if(alternative=="greater") {
    p_val <- 1 - pnorm(z_stat)
  }
  list(phat = phat, SE = SE, z_statistic = z_stat, p_value = p_val)
}
\end{lstlisting}

 

% Sample size for mean (sigma known)
\noindent\textbf{7. Sample Size for a Mean (known $\sigma$):} \\
$n=\left(\dfrac{z_{\alpha/2}\,\sigma}{E}\right)^2$.

 

 
\begin{lstlisting}
sampleSizeMean <- function(sigma, E, conf.level = 0.95) {
  alpha <- 1 - conf.level
  zcrit <- qnorm(1 - alpha/2)
  ceiling((zcrit * sigma / E)^2)
}
\end{lstlisting}

 

% Sample size for a proportion
\noindent\textbf{8. Sample Size for a Proportion (worst-case $p=0.5$):} \\
$n=\left(\dfrac{z_{\alpha/2}\cdot0.5}{E}\right)^2$.

 

 
\begin{lstlisting}
sampleSizeProp <- function(E, conf.level = 0.95) {
  alpha <- 1 - conf.level
  zcrit <- qnorm(1 - alpha/2)
  ceiling((zcrit * 0.5 / E)^2)
}
\end{lstlisting}

 

% Power for one-sample z-test
\noindent\textbf{9. Power for One-Sample z-Test (known $\sigma$):} \\
For $H_0:\mu=\mu_0$ vs. $H_A:\mu>\mu_0$,
\[
\text{Power} = 1-\Phi\!\left(\frac{\mu_0+z_{\alpha}\,\frac{\sigma}{\sqrt{n}}-\mu_{\text{alt}}}{\sigma/\sqrt{n}}\right).
\]

 

 
\begin{lstlisting}
powerZTest <- function(mu0, mu_alt, sigma, n, alpha = 0.05, alternative = "greater") {
  SE <- sigma / sqrt(n)
  if(alternative=="greater"){
    zcrit <- qnorm(1 - alpha)
    boundary <- mu0 + zcrit * SE
    power <- 1 - pnorm(boundary, mean = mu_alt, sd = SE)
  } else if(alternative=="less"){
    zcrit <- qnorm(alpha)
    boundary <- mu0 + zcrit * SE
    power <- pnorm(boundary, mean = mu_alt, sd = SE)
  } else {
    stop("For power, use alternative 'greater' or 'less'.")
  }
  power
}
\end{lstlisting}

 

% Process continuous joint PDF
\noindent\textbf{10. Process a Continuous Joint PDF:} \\
Total mass: $\displaystyle \iint_R f(x,y)\,dx\,dy=1$, \quad $E[X] = \int y\int x\,f(x,y)\,dxdy$, \quad $E[Y] = \int x\int y\,f(x,y)\,dydx$.

 

% Covariance and Correlation
\noindent\textbf{11. Covariance and Correlation:} \\
$\operatorname{Cov}(X,Y)=E[XY]-E[X]E[Y]$, \quad $\rho_{XY} = \dfrac{\operatorname{Cov}(X,Y)}{\sigma_X\,\sigma_Y}$.

 

 
\begin{lstlisting}
covCor <- function(X, Y) {
  list(covariance = cov(X, Y), correlation = cor(X, Y))
}
\end{lstlisting}

\end{document}
